\documentclass[12pt,a4paper]{article}

\usepackage[slovak]{babel}
\usepackage[utf8]{inputenc}
\usepackage{listings}
\usepackage{graphicx}
\usepackage{tabularx} 
\usepackage{amsmath} 
\usepackage{amssymb} 
\usepackage{hyperref} 
\usepackage{multicol} 

\lstset{
language=python
,breaklines=true
,basicstyle=\ttfamily
, showstringspaces=false}

\textwidth 6.5in
\oddsidemargin 0.0in
\evensidemargin 0.0in

\begin{document}

\thispagestyle{empty}
\begin{center}
    \large{
        \textbf{
            UNIVERZITA KOMENSKÉHO V BRATISLAVE \\ 
            FAKULTA MATEMATIKY, FYZIKY A INFORMATIKY
        }
    }
\end{center}

\vspace{2cm}

\begin{figure}[!h]
    \centering
    \includegraphics[width=3.5cm]{komlogo-new.pdf}
\end{figure}

\vspace{1cm}

\begin{center}
    \large{
        \textbf{
            ANALÝZA A RIEŠENIA PROBLÉMOV PORTÁLU data.gov.sk
        }\\
        
        2-INF-106	Informatika a spoločnosť \\
        (Seminárna práca)\\ 
        \href{https://github.com/koniiiik/opendata-sk-ias}{https://github.com/koniiiik/opendata-sk-ias} 
            
        \vspace{1.5cm}
        
        \textbf{
            Peter Csiba, Eduard Eiben, \\  
            Martin Kolínek, Michal Petrucha
        } \\
            %Analýza algoritmov učenia na báze zovšeobecnenej recirkulácie v obojsmerných neurónových sieťach

    }
\end{center}

\vfill

\begin{multicols}{2}
    \begin{flushleft}
        \textbf{23.5.2014}
    \end{flushleft}
    \begin{flushright}
        \textbf{Bratislava}
    \end{flushright}
\end{multicols}

\newpage

\section{Úvod}

Data.gov.sk, ďalej len {\bf portál}, je snahou o implementovanie princípov Open data \ref{opendata} v kontexte Slovenskej republiky. 

TODO - dopísať nakoniec 
TODO - motivácia - Open data is an emerging idea most developed in the UK and USA ... in SK its shit ... 
TODO - prejsť emaily od Jany, Jána, Hanyho a AFP. 

\subsection{Cieľ práce}
Našou prácou je zmysluplne pomôcť portálu data.gov.sk. Našimi kvalitami sú implementačné a analytické zručnosti. Jedným z problémov sú manuálne úkony na portály. Preto sa snažíme automatizovať žiadané procesy. Hlavnými dvoma procesmi sú automatická kontrola kvality dát a automatické získavanie relevantných datasetov, ktoré popisuje v časti \ref{problem-statement}. 

%\paragraph{Pôvodný cieľ} 
%Portál data.gov.sk obsahuje linky na dátové zdroje VS, avšak zatiaľ iba veľmi nepresne. Cieľom práce je aktualizovať údaje, ktoré sú tam uvedené (napr. url, kvalita) a najmä systematicky vyhľadať a doplniť informácie o ďalších dostupných údajoch.

\section{Postup práce}
Naša práca sa dá rozdeliť na dve časti. Prvou časťou bola \emph{analýza} problému, ktorý sme potom riešili \emph{implementáciou}. 

\subsection{Komunikácia}
Aby sme sa v tematike zorientovali, tak sme komunikovali s relevantnými subjektmi, ktoré sa problematikou priamo alebo nepriamo zaoberajú. 

\begin{itemize} 
  \item Úrad splnomocnenca vlády SR pre rozvoj občianskej spoločnosti - \emph{správca portálu} data.gov.sk .
  \item Aliancia Fair Play - zverejňujú verejné štátne dáta na ich portály datanest (\cite{datanest}), ktoré sú vhodné pre poukázanie korupcie verejných činiteľov. 
  \item OpenData.sk (\cite{opendata-sk}) - zaoberajú sa podobnou problematikou ako data.gov.sk. 
  \item Foaf.sk a Minio.sk - poskytujú rozhranie pre spracované verejné štátne dáta, ktoré je efektívne použiteľné.  
\end{itemize} 

\subsection{Stav portálu data.gov.sk} 
Po komunikácií a vlastnej analýze sme dospeli k záveru, že rozsah aj kvalita dát portálu sú značne obmedzené. K podobnému záveru prišli aj pracovné skupiny, ktorých výsledky sú zverejnené na opendata.sk. Navrhujú riešenia, ale zatiaľ nevidíme ich realizácie. Dôvody nedostatkov sú rôzne. Za hlavný dôvod považujeme, že portál data.gov.sk zhromažďuje len datasety, ktoré už sú zverejnené a chýba subjekt, ktorý ich aktívne žiada. 

Napríklad AFP získala užitočné datasety pomocou opakovaných infožiadosí, využijúc zákon, podľa ktoré sú štátne inštitúcie povinné zverejniť viaceré dáta. Avšak kým občanom nechýbajú, tak si inštitúcie nebudú vyrábať prácu navyše. Druhým príkladom spracovania verejných dát sú projekty foaf.sk a finstat.sk, ktorý vyriešili dostupnosť a spracovateľnosť verejných dát ich automatizovaným stiahnutím, offline spracovaním a ich zverejnením v jednoduchšej podobe spolu s relevantnými štatistikami. 

\subsection{Formulácia problému} 
\label{problem-statement}

Vzhľadom na obmedzený rozsah nášho projektu a prakticky neobmedzený rozsah práce na portály sme boli nútení si vybrať časť na ktorej budeme pracovať. Ostatné nápady sú popísané v časti \ref{future-work}. 

Vychádzali sme z dokumentu \cite{525} vypracovaným skupinou opendata.sk, ktorý obsahuje zoznam 525 datasetov, ktoré by bolo \emph{užitočné} zverejniť. Zvolili sme nasledovný postup:  
\begin{itemize} 
\item Exportovať zoznam do SQL DB a vyčištiť ho. 
\item Previazať databázu s data.gov.sk a zistiť, ktoré datasety už sú získané. 
\item Previazať s výsledkami manuálnej kontroly data.gov.sk z maraca 2014-03. 
\item Zautomatizovať manuálnu kontrolu z predošlého bodu, t.j. či tam dáta sú, aké majú chyby a tak ďalej. Inšpirovali sme sa jednoduchým skriptom od \cite{hany}. 
\item Pridať \emph{5-star} rating kvality datasetov. Podobná funkcionalita existuje v plugine \emph{ckanext-qa} pre CKAN, systém pre zverejnenie open data.  
\end{itemize} 

\subsection{Implementácia}

Najprv sme skonvertovali zoznam 525 užitočných datasetov do SQLite databázy $D$, ktorú sme následne prečistili. Napr. opravili niektoré preklepy, aby boli dva sémanticky ekvivalentné hodnoty rovnaké aj syntakticky. 

Využili sme CKAN REST API, aby sme získali aktuálny stav data.gov.sk. Vychádzajúc z API a databázy $D$ sme implementovali nasledovné dva skripty. 

\paragraph{check.py}
Keďže data.gov.sk obsahuje iba odkazy na datasety, je nutné kontrolovať, či sú odkazy platné. V tomto skripte k danému datasetu získame všetky linky a HTTP requestom overíme, či naozaj existuje. Navyše kontrolujeme hlavičky requestu a snažíme sa uhádnuť reálnu príponu súboru, ktor8 sa môže líšiť. Ďalej sa snažíme zistiť, či išlo o redirect a či stránka obsahuje formulár, ktorý je rozhraním pre dataset. Na základe týchto údajov aproximujeme kvalitu datasetu podľa štandardnu popísaného v časti \ref{zakon-kvalita}. 
%TODO? K tomuto Edovmu iba follow-up, že dva z tých troch datasetov riešim ja spolu s maintainermi (zle implementujú HTTP), tretie je iba zle zadaná URL v CKANe.

\paragraph{vyhladavac.py} 
Tento semi-automatický skript sa snaží prepojiť záznamy databázy $D$ so záznammi na data.gov.sk. Pre každý záznam databázy $D$ vyhľadá pomocou CKAN API datasety s podobným názvom. V prípade, že výsledkom je viacero, tak skript ponúkne všetky možnosti. Tieto prepojenia sú ukladané do našej databázy $D$. 

Niekedy sa stane, že v CKANe daný dataset skutočne je, ale skript ho nenájde. Napr. môže byť rozdelený na viacero CKANových datasetov. Takéto prípady sme označili za neexistujúce a zapísali do \href{https://github.com/koniiiik/opendata-sk-ias/blob/master/problematicke_id}{problematicke\_id} spolu s popisom, čo sa pokazilo. Tieto prípady sa dajú vyriešiť aj ručne keďže ich je pomerne málo. Nakoniec sme sa neexistujúce prípady snažili priradiť ručne. Z databázy zaujímavých sme zistili pod ktoré ministerstvo patria a použili ako filter na data.gov.sk.

%Ja som práve pushol upravenú verziu môjho skriptu spolu s výsledkami pre viac ako polovicu toho XLS zoznamu. Viac už fakt takto v noci nevládzem. Aj by som tento mail hodil na wiki, ale ani na to si u6 netrúfam; feel free hodiť to tam zajtra.

%Ak sa niekomu chce pokračovať a dochrúmať zvyšok, stačí naimportovať do databázy ten dump, čo som zavesil, a pustiť skript. Potrebuje to Pythoniu knižnicu requests, zvyšok by mala byť štandardná knižnica.
Zaujímavosťou boli problémy s názvami datasetov. Napr. dvakrát sa stalo, že bolo viacero datasetov, ktorých názvy sa líšili iba v posledných pár slovách a CKAN search na jeden dataset vrátil všetky okrem toho hľadaného\footnote{
Napr. dvojica 313, 314 a jedna trojica 1716, 1717, 1718.
}. Alebo niektoré datasety mali rozdielne názvy na data.gov.sk a v databáze $D$, napriek tomu, že význam bol ten istý. 

\section{Výsledky}

TODO - počkať na Maťa 


TODO (must!): koľko \% tých datasetov, čo mali byť zverejnené, tam skutočne je (teda tých zverejniteľných)

%\subsection{Vyhľadávanie ďalších datasetov}
%TODO zopar sample queries na vsetky, resp. konkretne z 525

\section{Záver} 

TODO - na koniec 

\subsection{Budúca práca} 
\label{future-work}

Zobrať názvy z \cite{525} a vyhľadať chýbajúce datasety na slovenskom Internete (nad rámec rozsahu našej práce). 

Vygenerovať zoznam štátnych inštitúcií a automaticky vyhľadávať potenciálne datasety (nad rámec rozsahu našej práce).

TODO niečo ďalšie 


\renewcommand{\refname}{Literatúra}
\phantomsection
\addcontentsline{toc}{section}{Literatúra}
\begin{thebibliography}{99}
  \bibitem{proj} \url{https://github.com/koniiiik/opendata-sk-ias} - Repozitár nášho projektu s dokumentáciou. %TODO - ensure only zverejnitelne
  \bibitem{525} \url{http://www.otvorenavlada.gov.sk/datasety-statnej-spravy/} - Zoznam datasetov \emph{užitočných} na zverejnenie.
  \bibitem{hodnotenie} \url{http://www.otvorenavlada.gov.sk/navrh-hodnotiacej-spravy-iniciativy-pre-otvorene-vladnutie-v-slovenskej-republike/} - Návrh hodnotiacej správy Iniciatívy pre otvorené vládnutie v Slovenskej republike.
  \bibitem{kvalita} \url{http://www.zbierka.sk/sk/predpisy/55-2014-z-z.p-35621.pdf} - Výnos 55/2014 aj o kvalite Open data. 
  \bibitem{hany} \url{https://github.com/hanecak/data.gov.sk-link-check/} - Základná kontrola odkazov datasetov portálu CKAN. 
  \bibitem{datanest} \url{http://datanest.fair-play.sk/pages/index} - Datanest Aliancie Fair play. 
  \bibitem{opendata-sk} \url{http://opendata.sk/liferay/studia-open-data-portal} - OpenData.sk. 
  \bibitem{opendata-wiki} \url{http://en.wikipedia.org/wiki/Open_data} - Definícia open data na Wikipédií. 
\end{thebibliography}


\section*{Príloha - Vysvetlenie pojmov} 
\subsection*{Open data} 
\label{opendata} 
TODO prelozit 
Open data je myšlienka, že niektoré dáta by mali byť dostupné pre každého na akékoľvek použitie, bez akýchkoľvek licencií, patentov alebo iných kontrolných mechanizmov. Ciele open data sú podobné ako iných \emph{otvorených} spoločností, ako sú open source, open hardware a iné. Filozofia open data je stará ale termín \emph{open data} je súčasný a naberá na popularite s rastom Internetu a WWW. Najpokričlejšími portálmi sú data.gov a data.gov.uk (\cite{opendata-wiki}). 

\subsection*{Hodnotenie kvality datasetov}
\label{zakon-kvalita} 
Kvalitu dát sme hodnotili podľa \emph{výnosu 55/2014} (\cite{kvalita}) popisujúci aj \emph{kvalitu datasetu poskytovaného povinnou osobou}, ktoré je rozdelené na šesť úrovní: 
\begin{itemize} 
\item úroveň 0, pri ktorej nie je dataset poskytovaný
v elektronickej forme,
\item úroveň 1, pri ktorej je dataset dostupný vo webovom
prostredí,
\item úroveň 2, pri ktorej je splnená požiadavka uvedená
v písmene b) a obsah datasetu je štruktúrovaný tak,
že umožňuje automatizované spracovanie,
\item úroveň 3, pri ktorej sú splnené požiadavky uvedené
v písmene c) a dataset je poskytovaný v otvorenom
formáte, nezávislom na konkrétnom proprietárnom
softvéri,
\item úroveň 4, pri ktorej sú splnené požiadavky uvedené
v písmene d) a na identifikáciu údajov datasetu a ich
vzťahov sa používajú refencovateľné identifikátory,
\item úroveň 5, pri ktorej sú splnené požiadavky uvedené
v písmene e) a dataset a jeho interné a externé
vzťahy majú charakter identifikátormi prepojených
údajov.
\end{itemize} 

Ak sa údaje poskytujú pre automatizované spracovanie, štandardom kvality datasetu poskytovaného
povinnou osobou je aj ich poskytovanie ako datasetu s otvorenými údajmi podľa § 53 a v kvalite najmenej úrovne 3.


\end{document} 
